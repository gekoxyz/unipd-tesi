%!TEX root = ../main.tex
\begin{abstract}
Questa tesi analizza il design e i principi operazionali di Starlink, una rete di satelliti in orbita LEO sviluppata da SpaceX per fornire internet a bassa latenza e alta velocità globalmente.
Con una costellazione di oltre 7000 satelliti nel 2024, Starlink rappresenta un grande passo in avanti per le comunicazioni satellitari, superando le limitazioni tradizionalmente associate con sistemi in orbita Geostazionaria.
La tesi analizza i segmenti utente, di terra e spaziali ponendo particolare attenzione all'antenna Phased Array, i collegamenti laser inter-satellitari e le tecniche di modulazione utilizzate.

Lo studio rivela che l'utilizzo di una rete satellitare in orbita LEO riduce significativamente la latenza e permette di espandere la connettività alle aree remote o non servite.
Starlink fornisce informazioni utili sul futuro di Internet via satellite, mettendo un precedente per altri progetti simili.
\end{abstract}