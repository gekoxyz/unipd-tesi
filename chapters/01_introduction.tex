%!TEX root = ../main.tex

\chapter{Tipi di costellazioni satellitari}
\label{chp:intro}

Una costellazione di Internet via satellite è una costellazione di satelliti artificiali che forniscono servizi di Internet via satellite. In particolare, il termine si riferisce a una nuova generazione di costellazioni molto grandi (a volte indicate come megacostellazioni) che orbitano nell'orbita terrestre bassa (LEO) per fornire servizi Internet a bassa latenza e ad alta larghezza di banda (banda larga) \cite{jose_del_rosario_nsr_2018}.
Nel 2020, il 63\% delle famiglie che vivono in zone rurali del mondo non ha accesso a Internet a causa dei requisiti infrastrutturali dei cavi sotterranei e delle torri di rete. Le costellazioni Internet via satellite offrono una soluzione a basso costo per espandere la copertura \cite{makena_young_low_2022}.

\section{Vantaggi della Low Earth Orbit}

I satelliti sono tipicamente lanciati in una di tre orbite: Low Earth Orbit (LEO) dai 160 ai 2'000 km, Medium Earth Orbit (MEO) dai 2000 ai 35786 km, o Geosynchronous Orbit (GEO) a 42164 km.
Ciascuna delle tre orbite ha i suoi vantaggi.
Per esempio, una costellazione in GEO può avere copertura globale con solo tre satelliti per la sua distanza dalla superficie terrestre.
La GEO è popolare per le comunicazioni per questo motivo.
Però, dato che i satelliti sono così distanti dalla Terra la latenza è molto alta.

Le nuove generazioni di internet via satellite stanno collocando i satelliti in LEO invece della tradizionale GEO per una serie di motivi.
Infatti i satelliti lanciati in LEO:
\begin{itemize}
  \item sono tipicamente più piccoli e più leggeri di quelli in GEO, quindi serve meno carburante per mandarli in orbita e in generale sono ridotti i costi di lancio.
  \item dato che i satelliti sono più vicini i terminali utente possono rilevare più satelliti allo stesso tempo e conettersi quindi al satellite più conveniente \cite{makena_young_low_2022}.
\end{itemize}

% https://en.wikipedia.org/wiki/SpaceX#Starlink_2


